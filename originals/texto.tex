\documentclass[a4paper,11pt,openright,openbib]{article}
\usepackage[portuges]{babel}
\usepackage[T1]{fontenc}
\usepackage{ae}
\usepackage[utf8]{inputenc}
\usepackage[pdftex]{graphicx}
\usepackage{url}
\usepackage{listings}
\usepackage{verbatim}
\usepackage{enumerate}
\usepackage[a4paper, pdftex, bookmarks, colorlinks, linkcolor=black, urlcolor=blue]{hyperref} 
\usepackage[a4paper,left=2.5cm,right=2.5cm,top=3.5cm,bottom=3.5cm]{geometry}
\usepackage{colortbl}
\usepackage[margin=10pt,font=small,labelfont=bf]{caption}
\usepackage{mdwlist}


\setlength{\parindent}{0cm}
\setlength{\parskip}{2pt}



\title{
\begin{center}
	\begin{tabular}{l c r}
	\Large{\textbf{Criticas de Cinema}} & \\
	\end{tabular}
	\end{center}
}

\begin{document}

\maketitle

\pagestyle{headings}
\pagenumbering{arabic}
\newpage
\tableofcontents
\newpage

\section{Sombra da noite - 2012}

O oitavo filme da dupla Tim Burton e Johnny Depp, baseado na série Dark Shadows, exibida pela rede norte americana ABC entre junho de 1966 e abril de 1971, é uma mistura de tudo que o cineasta tem de melhor e pior. O resultado é uma engraçada (admito) sequência de situações embaraçosas e desconexas com potencial também para decepcionar até os maiores defensores do diretor.

O longa começa com uma apresentação de cinco minutos da saída da família Collins de Liverpool, no século 18, e sua ascensão como potência da indústria pesqueira em Collinswood, Estados Unidos. Seu filho, Barnabás (Depp), acaba se relacionando com Angelique (Eva Green), uma empregada da família que é também uma bruxa. 

Quando o jovem se nega a casar com a moça e escolhe Josete (Bella Heathcote) para ser sua noiva, ele é amaldiçoado e aprisionado por dois séculos. Acorda em 1972 e deve enfrentar a decadência de sua família em mundo moderno que não compreende. 

Embora a premissa seja boa, a produção sofre com dois graves problemas: falta de uma narrativa consistente, o que torna difícil se envolver com a história, e a necessidade de conhecer a temática da série Dark Shadows e um pouco da cultura pop dos anos 1970 para entender os melhores momentos do longa, cujo ápice é o show de Alice Cooper, o roqueiro que o vampiro simplesmente não entende que é um homem.

O interessante é que a trama dá muita importância para Victoria Winters (Bella Heathcote novamente) no começo, mas seu sumiço no meio do filme deixa claro que o roteiro de Seth Grahame-Smith, autor do livro Orgulho e Preconceito e Zumbis, que mostraria a visão da moça, foi deixado de lado em algum momento em prol do que Burton queria – Barnabás como estrela. 

Além disso, os personagens não são cativantes, falta charme e, às vezes, até coerência à produção. São tantos elementos misturados que parece que Burton estava na duvida se fazia um terror gótico, uma comédia infantil, um longa de humor negro, um drama de superação ou simplesmente algo estranho. Pelo menos numa coisa ele acertou: a direção de arte, claro.

Sim, o visual é o ponto alto de Sombras da Noite. As cores fortes foram muito bem escolhidas, os figurinos e ambientes de cena tem o estilo Burton e são marcantes, cortesia de Rick Heinrichs, parceiro de longa-data do cineasta e ganhador do Oscar por Sleepy Hollow- A Lenda do Cavaleiro Sem Cabeça.

Diferente de outras grandes atuações de Depp, como em Swenney Todd, o ator não consegue tanto impacto na hora de dar vida ao bizarro vampiro do século 18. Isso não o impede de se divertir horrores no papel da estranha e burra criatura, que cai uma, duas, três vezes nas armações de Angelique – a única personagem com algum senso de propósito na trama – mesmo que deturpado.

Sombras da Noite é uma coleção de momentos bons e ruins sem muita coerência entre si. Uma produção leve com potencial para se dar bem na Sessão da Tarde daqui a alguns anos.

\section{Febre do Rato}
Cláudio Assis é um cineasta criativo e ousado que se deixa levar pelas ideias. Sabe que a hesitação é apenas a antessala da mediocridade e mesmice e aposta alto num cinema criativo que se mostra um verdadeiro oásis em meio ao deserto de fórmulas recicladas e batidas que virou o cinema nacional atual. 

Seu novo longa, Febre do Rato, é passional, intenso e brilhante ao contar a história de Zizo (Irandhir Santos, impecável), um poeta marginal e libertário da grande Recife. Ele publica o tabloide anarquista que dá nome ao filme, escreve poesia para os amigos - como o casal Pazinho (Mateus Nachtergaele) e a travesti Vanessa (Tânia Moreno) – e faz sexo com Stellamaris (Maria Gladys) e Anja (Conceição Camaroti), mulheres distantes dos padrões impostos pela ditadura da beleza. Suas convicções de artista inconformista ameaçam ruir quando cruza o caminho de Eneida (Nanda Costa, de Sonhos Roubados), jovem de espírito livre que resiste às suas investidas e mexe com sua vaidade. 

A história de Zizo e seus amigos marginais é ambientada numa Recife belamente fotografada em preto-e-branco por Walter Carvalho, palco de típicos problemas urbanos, como favelas rivalizando com arranha-céus opulentos e poluição e barulho sufocando a vida de seus moradores. O poeta, ao divulgar cada nova edição do seu periódico, grita pelos alto-falantes de uma velha variant: "Vocês sabem o barulho que essa cidade tem. Vocês sabem o gosto. O cheiro." 

Apesar do engajamento de Zizo reverberar na tela, o filme não tem um tom panfletário, muito em função do equilíbrio de elementos do roteiro de Hilton Lacerda. É nítido que o filme foi trabalhado em seus detalhes para compor um todo hamonioso. Cláudio Assis não confunde anarquia temática com avacalhação no processo de produção. Tudo no filme é muito bem cuidado e lapidado. "É preciso ter respeito com o dinheiro público", disse Assis no último Cine Ceará, do qual saiu eleito melhor diretor. 

Existem inúmeras cenas ousadas e outras que transitam no limite do escândalo no longa, todas muito bem encenadas, enquadradas e de um furor dramático impressionante. Uma sequência explícita filmada de cima de um ménage à trois consegue impactar mais por sua beleza que pela ousadia. A crueza de genitálias expostas se transforma em poesia na mão do artista Cláudio Assis. 

Febre do Rato é apenas seu terceiro longa-metragem, mas no atual cenário nacional não há outro cineasta com uma cinematografia tão pungente. Ninguém ficou indiferente a Amarelo Manga, seu primeiro longa, lançado há dez anos no Festival de Brasília, nem a Baixio das Bestas, que veio cinco anos depois. Com Febre do Rato o incômodo e a surpresa se repetem no espectador. 

Mais do que um tipo inquieto e aguerrido, Assis volta a se afirmar como cineasta que domina plenamente o ofício. Seu filme lança a isca intelectual de que a arte pode servir como meio de libertação psicológica e criativa, uma válvula de escape e uma frente de guerrilha ao cerceamento das liberdades, ao lugar-comum e ao pensamento único que padroniza as vontades. E cinema ainda é arte, afinal. 

\section{Prometheus}
O retorno de Ridley Scott ao gênero que ajudou a definir no cinema era algo muito esperado pelos fãs desde 1982, quando lançou seu último filme de ficção-científica Blade Runner e nada poderia ser mais apropriado do que um prólogo de Alien. Com boa parte da tensão e visual que estamos acostumados a ver em produções anteriores do cineasta, Prometheus é um filme ambicioso que até pode ser considerado o Alien do século 21, porém isso não significa que ele cumpra a expectativa criada a sua volta.

A questão é que o longa é ambicioso demais e acaba sem conseguir abarcar tudo o que pretende. Ele trata da procura por respostas sobre a criação da humanidade e isso nos leva a uma lua distante, a bordo da nave que dá nome ao longa-metragem, onde nos deparamos com momentos de terror e mortes violentas. É exatamente a tentativa de misturar essas profundas questões com a obrigação de dar sustos e ter cenas chocantes gratuitas que impede o filme de ser melhor do que é.

O interessante é que desde o início Prometheus procura se distanciar de Alien (é importante saber disso ao ir ao cinema), porém recicla muitos elementos que deram certo no clássico de 1979 e, sem vergonha de beber da mesma fonte, constrói aos poucos uma atmosfera de tensão e paranoia. Só que em nenhum momento a coisa fica tão séria quanto a bordo da Nostromo (nave onde Ripley era tenente), pois as motivações dos personagens são claras demais quando não deveriam e obscuras quando não precisavam ser – o que dá um tom artificial em alguns momentos – principalmente porque Alien já fez tudo isso antes.

No entanto, isso não impede que tenhamos atuações sólidas e convincentes. Michael Fassbender rouba a cena mais uma vez, como já havia feito em X–Men: Primeira Classe. O ator interpreta o androide David de uma forma assustadora e bizarra, sem sentimentos, ao ponto de incomodar o espectador que observa, passivo, decisões importantes serem tomadas por uma máquina sem medo e sem consideração pela vida humana. 

A personagem de Noomi Rapace (Os Homens Que Não Amavam As Mulheres), a arqueóloga Elizabeth Shaw, não chega aos pés da eterna tenente Ripley (Sigourney Weaver), porém consegue conquistar o público com seu jeito sonhador, aparentemente frágil e muito mais feminino. Charlize Theron (Branca de Neve e o Caçador) mais uma vez está bem, desta vez no papel da inescrupulosa e bela Vickers. Uma pena que a executiva da Weyland Corp. seja um tanto dispensável para a trama – se melhor explorada, seria capaz de contribuir para o clima de paranoia, tão importante em Alien - O Oitavo Passageiro.

É claro que o grande astro do filme é o show de imagens, sons, luzes e efeitos. Os cenários são realistas e detalhados, a sala com a cabeça humana gigante dá calafrios e o clima de Alien aparece com força em alguns deles – ponto para o quesito nostalgia. É impressionante a forma com que os motores da nave Prometheus fazem o cinema tremer – a intensidade é assustadora. As criaturas e, consequentemente, as cenas de sustos são exatamente como esperamos – coisa de quem sabe o que está fazendo. Até o 3D vale o ingresso.

Quem curte sci–fi vai perceber que a ciência ficou de lado e a tecnologia – muito mais clean e avançada do que no filme que se passa 30 anos no futuro – funciona apenas como ferramenta narrativa, sem muito embasamento, mesmo dentro do gênero. Os problemas aparecem até durante a simples analise do material genético de um extraterrestre, cujo resultado não é convincente, e olha que não sou especialista em DNA. No final das contas parece ser uma forma gratuita de negar as teorias de Charles Darwin e justificar o US\$ 1 trilhão (que em 90 anos nem será tanto assim) investido na construção da nave.

Como fã de ficção científica, é impossível não gostar de Prometheus, um dos poucos bons filme do gênero dos últimos anos, que teve alguns destaques como o suíço Cargo, A Origem e Star Trek. Entretanto, é inegável que a aventura futurista de Ridley Scott patina um pouco do meio para o final, é previsível e, às vezes, dá a sensação de que existiam certas obrigações (impostas pelo estúdio?) a serem cumpridas – como a explosão de uma criatura nojenta saindo do peito de alguém. 

Ainda assim, seu magnífico visual, boa trama e atuações fazem do longa uma obra que vale a pena ser vista. Só não vá ao cinema esperando assistir a algo tão impactante como Alien – O Oitavo Passageiro.

\section{Para Sempre}
Drama romântico inspirado em uma história real, Para Sempre começa carregando nas tintas. Além do exagero numa cena de acidente de carro, todo início do filme é contaminado por um romantismo tão excessivo que chega a enjoar. Não precisava tanto para fazer o público entender que Leo (Channing Tatum, de Anjos da Lei) e Paige (Rachel McAdams, de Meia Noite em Paris) formam um casal muito apaixonado. Pelo menos até o acidente que muda suas vidas.

Leo é um jovem empreendedor, dono de um pequeno estúdio de gravação. Apesar do ramo não ter muito futuro, já que qualquer um pode fazer uma boa gravação com um computador, ele acredita na sua paixão pela música. Já Paige largou a faculdade de Direito para estudar arte e tem se saído bem como uma escultora de talento promissor.

No entanto, tudo muda quando eles sofrem um acidente de carro. Leo sai com poucos ferimentos, Paige, contudo, bate a cabeça e fica alguns dias em coma. Quando ela recobra a consciência, não reconhece o marido, nem se lembra de nada dos últimos cinco anos de sua vida. É quando surgem seus pais, com quem ela não falava há anos, desde que abandonou a faculdade e foi viver de sua arte. O porquê dela ter rompido com a família, com o antigo noivo e com um estilo de vida totalmente diferente estão perdidos nos anos que sua memória insiste em não lembrar.

Mais do que não lembrar, ela está diferente. Gostos, valores e atitudes são de uma Paige antiga, que Leo não conheceu e não reconhece em sua esposa. E nem ela a ele.

É a partir desse conflito que o filme cresce, diminuindo sua voltagem romântica e abordando o drama de Leo, que tenta restabelecer sua vida ao lado da esposa. Mas, por mais que ela se esforce, ele continua sendo um estranho, assim como a vida que eles viviam e os amigos que tinham. Um problema que se agrava com a interferência da família dela, que sempre desaprovou a vida que decidiu levar e vê agora a oportunidade de recuperar a filha que tinham perdido.

Na composição desse drama, Para Sempre não escapa de alguns clichês, mas acerta em evitar muitos outros. Em alguns momentos, parece levar a trama para um caminho previsível, porém desvia-se a tempo de não cair no óbvio. Na comoção do marido apaixonado que tenta reconquistar a esposa, a cara de cãozinho pidão de Channing Tatum ajuda bastante, conseguindo até alguns lampejos de boa atuação.

Mesmo sendo um tanto irregular, o filme funciona bem como história romântica. Vai até um pouco além e traz uma subtrama de reconciliação, atenuando maniqueísmos e diminuindo os tons estereotipados de alguns personagens. Nesta subtrama, a participação da atriz veterana Jessica Lange como mãe de Paige presenteia o espectador com um momento de intensidade e ótima interpretação. São pequenas surpresas, ou momentos, como gosta de dizer o protagonista Leo, que fazem o resultado final ser melhor que a soma das partes.

\section{Os acompanhantes}
Há algo de sedutor na improvável amizade que Os Acompanhantes constrói. Um tipo de charme antiquado, de comédia construída mais pela atuação do que pelas situações. Completa esse charme – atribuindo-lhe certa beleza – o processo de aceitação a que leva a amizade: aceitação de si e aceitação do outro. Um delicado processo de reconciliação nascido da relação entre figuras solitárias e diferentes. Tudo isso moldado pelo humor cuja graça não depende do ridículo e vem sempre acompanhado de uma estranha ternura.

Luis Yves (Paul Dano, de Pequena Miss Sunshine) é um jovem professor de literatura que, após um constrangedor incidente na universidade onde leciona, se muda para Nova York. Tímido e de fala cordial, Yves é perseguido por algumas fantasias sexuais que o deixam bastante embaraçado. Na cidade grande, vai dividir o apartamento com o excêntrico Henry Harrison (Kevin Kline, de O Clube do Imperador), um aposentado de discurso assertivo e ideias e hábitos bastante incomuns.

Enquanto consegue um emprego em uma revista sobre meio ambiente, Yves vai descobrindo uma das atividades de seu colega de apartamento. Harrison é o que ele próprio denomina como um “homem extra”, um acompanhante frequentemente solicitado por ricas e solitárias senhoras da alta sociedade.

Mas ele não é um gigolô, pois não cobra por sua companhia. Ele o faz apenas interessado nos bons restaurantes, bons vinhos e bons programas que estas senhoras podem proporcionar. Mais do que um “homem extra” simplesmente, ele se julga um acompanhante essencial, cujos modos, cordialidade e atenção com suas damas o colocam num nível que transcende a mera companhia. Ao menos é como ele se vê.

Dentro da convivência entre eles, surgirá uma relação bastante incomum de companheirismo. Quase sem perceber, Yves se torna um aprendiz de seu amigo acompanhante, ao mesmo tempo que se apaixona por uma colega de trabalho (Katie Holmes, de O Casamento do Meu Ex) e também tenta resolver suas fantasias secretas.

De ritmo bem cadenciado, Os Acompanhantes é uma comédia que aposta na delicadeza do humor. Sua graça nasce do excêntrico, mas não se apoia na esquisitice de seus protagonistas, como se fossem meras muletas para o riso. Seu humor está no carisma que esses personagens transpiram dentro de suas excentricidades. Está também na interpretação inspirada dos atores, com destaque para Kevin Kline.

Não se pode deixar de citar a participação de John C. Reilly (de Precisamos Falar Sobre o Kevin) como Gershon Gruen, um mecânico que trabalha no metrô da cidade e costuma pedalar nas horas vagas. Ele mora no mesmo prédio que Yves e Harrison e terá um papel importante – e não menos delicado e divertido – no desenlace da fita.

Em Os Acompanhantes a reconciliação e a aceitação são como o desenlace de uma fábula, na qual a amizade entre solitários faz deles pessoas melhores.

\section{Branca de Neve e o Caçador}
Há grande diferença entre divertir o público e fazer pouco caso dele. Esta última situação vem se tornando corriqueira em algumas produções hollywoodianas, que confundem cinema de puro lazer com desleixo com o processo de criação e produção. Branca de Neve e o Caçador, felizmente, não se encaixa neste perfil. Tem trama ao alcance de todos, é nitidamente feito para agradar a públicos diversos, mas conta com produção irrepreensível e roteiro bem alinhavado que leva o espectador a embarcar na trama.

A famosa fábula dos irmãos Grimm, eternizada na animação da Disney, ganhou duas versões live action este ano. A primeira, intitulada Espelho, Espelho Meu – com Julia Roberts e Lily Collins como protagonistas-, tinha a pretensão de recriar o conto em tom de comédia. Não deu certo. Crítica e público a rejeitaram solenemente. A proposta de Branca de Neve e o Caçador foi pelo caminho da aventura épica. Funcionou. Muito pelo bom trabalho da produção, que soube equacionar os diversos elementos que compõem um blockbuster.

Era preciso um chamariz para atrair o grande público, o que levou Kristen Stewart – a estrela de Crepúsculo - ao papel de Branca de Neve. Solução por um lado, Kristen era problema de outro: como compensar o limitado arsenal dramático da atriz? O trio de roteiristas fez a compensação valorizando os coadjuvantes da trama e apostando alto na vilã, que, como se sabe, na história de Branca de Neve tem importância equivalente à protagonista. Charlize Theron, com talento pra ela e para doar aos necessitados, foi a escolha mais que acertada da produção para o papel de Rainha Má. 

Direção de arte e figurino são outro show a parte. David Warren, responsável pela direção de arte do premiado A Invenção de Hugo Cabret, assina um trabalho impecável e detalhista, responsável pela ambientação obscurantista de um reino mergulhado nas trevas pela malvada monarca. Por outro lado, Warren resgata com competência o tom de conto de fadas da história quando Branca de Neve chega à Floresta Encantada. Tudo com riqueza de detalhes impressionante. A vencedora do Oscar por Alice no País das Maravilhas, Collen Atwood, reforça o trabalho idealizando um figurino que remete ao lúdico, mas não se distancia do real. 

O estreante diretor Rupert Sanders entrega uma versão muito mais sombria e violenta da história consagrada pela Disney. Na trama, o caçador Eric (Chris Hemsworth, o Thor) é contratado pela perversa rainha Ravenna (Theron) para encontrar Branca de Neve. A rainha precisa devorar o coração da jovem para, assim, conseguir a beleza eterna. O caçador, claro, percebe estar do lado errado da luta, se encanta pela heroína e passa a integrar a resistência contra as pretensões malignas da rainha. Os anões, agora oito, não podiam faltar e entram na trama para ajudar Branca a recuperar o trono.

A presença de Kristen Stewart estampada no cartaz do filme pode gerar reações diametralmente opostas. Ao passo que é capaz de atrair os adolescentes fanáticos pela Saga Crepúsculo, pode afugentar os detratores da franquia teen. Se o leitor faz parte deste último grupo, tire a bobagem vampiresca da cabeça e curta Branca de Neve e o Caçador. Diversão garantida para toda a família sem afrontar a dignidade de ninguém. 

\section{Battleship - A batalha dos Mares}
Existem duas maneiras de prender a atenção dos espectadores numa sessão de cinema por duas horas. Ou se segue o método tradicional, meio fora de moda hoje em dia, de contar uma boa história com personagens interessantes, ou se faz muito barulho. Neste último caso, conta-se com a mãozinha providencial dos modernos sistemas de som dos multiplexes. Battleship – A Batalha dos Mares se enquadra neste último caso: uma bobagem repleta de clichês e diálogos tolos que mantém seus sentidos entorpecidos à força de muita reverberação. 

Não dava para esperar coisa melhor. A produção é dirigida por Peter Berg, o mesmo diretor de obras "relevantes" como Hancock, Bem-Vindo à Selva e O Reino. Se não bastasse, o marketing do filme o anuncia como a nova empreitada dos produtores de Transformers, como se isso fosse motivo de orgulho. 

Não à toa A Batalha dos Mares é o que é: uma perda de tempo sem nada a acrescentar. A trama, ou a falta dela, fala de uns alienígenas que resolvem responder a uma mensagem humana enviada ao espaço com fogo. Chegam à Terra botando para quebrar, com o objetivo de assumir o controle e nos despejar do planeta. Para salvar o mundo do seu fatídico fim, o arremedo de roteiro de Erich e Jon Hoeber empurra goela abaixo do público uma série de personagem tão rasos e tolos que fica difícil de acreditar que seremos salvos por eles. 

Neste quesito quem se destaca é o mocinho da história, o tenente Hopper (Taylor Kitsch, de John Carter). Na verdade, até agora não sei qual é sua patente de fato. Num determinado momento do filme é primeiro-tenente, em outro é capitão de corveta, ou seja, duas patentes acima. Mas nem vou me ater aos furos que o filme dá nesta área militar porque daria para encher uma página. Para se ter uma ideia, por exemplo, o navio do herói é um contratorpedeiro, embarcação militar guarnecida por uns quinhentos homens em média. No filme parece que a tripulação não tem mais de 20 militares de tão mal feita que é a ambientação. 

Como o longa segue à risca a cartilha dos lugares-comuns e estereótipos, temos como o alvo romântico do tenente-capitão Hopper a modelo Brooklyn Decker, de beleza inversamente proporcional a seu talento de atriz. E olha que a mulher é, de fato, muito bonita. Ela é a filha do almirante Shane, interpretado por Liam Neeson, que não sei responder o que faz nesse filme. A cena na qual o casal se conhece, que abre o longa, é talvez a coisa mais constrangedora e nosense que vi nas telas nos últimos anos. 

Tenho de admitir que o filme me fez rir algumas vezes, mas não exatamente nos momentos em que tentou ser engraçado. A risada veio com alguns diálogos surreais e lamentáveis. Como não rir quando o mocinho, diante do ataque alienígena que já afundou duas embarcações da Marinha, matou seu irmão e está prestes a fazer repousar no fundo do mar seu próprio navio, diz: “Estou com um mau pressentimento”. Hã? Não falta nem aquele clichê batido dos dois militares na iminência de uma ação suicida: “Foi uma honra lutar com você”.

A cantora Rihanna faz uma ponta no longa como uma sargento, mas nem dá para avaliá-la porque pouco faz ou tem a dizer. Os efeitos especiais são muito bons, como de hábito, mas não capazes de disfarçar a história sem estofo. Deve agradar o público adolescente acostumado a ouvir seus ipods no volume máximo, mas dificilmente convencerá um público mais adulto que já anda de saco cheio de ver mais do mesmo. 

\section{Men in black - Homens de Negro}
Lançado em 2002, Homens de Preto 2 ficou muito aquém do original de 1997 e decepcionou fãs do bem-humorado sucesso estrelado por Tommy Lee Jones de Will Smith. Essa segunda sequência tenta regastar a afinada dinâmica dos carismáticos agentes J (Smith) e K (Jones) e, de fato, consegue ser bem mais interessante e divertido que seu antecessor.

É bem verdade que Homens de Preto 3 demora a engatar, taxia por longos quinze minutos até decolar de fato. Fica nítido que se poderia enxugar uns bons minutos do filme sem nenhum prejuízo ao desenvolvimento da trama. No entanto. quando finalmente entra nos trilhos, a produção compensa o preâmbulo extenso demais e o universo divertido dos homens de terno preto invade a tela. 

O filme começa com a fuga de Boris, o Animal (que, aliás, odeia ser chamado assim) de uma prisão lunar onde os humanos mantêm os criminosos alienígenas de maior periculosidade. Depois de 40 anos encarcerado, ele não está nenhum pouco feliz e quer vingança contra o agente K, responsável por sua prisão, amputação de um de seus braços e aniquilação de seus planos de invadir a Terra. 

Para mudar os rumos dos acontecimentos, Boris volta ao passado para eliminar K e dar continuidade a seus intentos dominadores, o que obriga J a fazer o mesmo e tentar impedi-lo. E é na década de 60 que Homens de Preto 3 tem seus melhores momentos. Os paradoxos temporais gerados pela viagem no tempo são interessantes e multiplicam as questões dramáticas, gerando momentos engraçados e de puro suspense. 

A ambientação serve de inspiração a muitas tiradas hilárias, que envolvem racismo, movimento hippie e contracultura. Até o ícone da pop art Andy Warhol dá o ar da graça e é responsável por um dos melhores momentos do longa. K, em sua versão mais jovem e não menos taciturna, é interpretado pelo ator Josh Brolin que, de alguma forma, consegue fazer um Tommy Lee Jones melhor que Tommy Lee Jones. Este, por sua vez, não aparece em mais que dez minutos de projeção. O filme conta também com a participação especial de Emma Thompson como O, a nova chefe da agência especializada no controle de alienígenas cuja história se mistura com a do agente K e envolve um segredo do passado . 

Com Barry Sonnenfeld mais uma vez no comando e um quarteto de roteiristas mais inventivo, o longa resgata o clima do primeiro filme e chega a flertar com o emocional em seus momentos finais. As criaturas estranhas e maravilhosas, marca registrada da série, estão presentes, mas, como no longa original, são os personagens humanos os destaques.Não dá pra dizer que a franquia ganhou um novo fôlego, mas certamente se redimiu da insossa primeira sequência. A dica é "neuralizar" o segundo episódio e se divertir com Homens de Preto 3. 

\end{document}